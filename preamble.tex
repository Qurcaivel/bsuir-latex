\documentclass[a4paper, oneside, 14pt, final]{extreport}

\usepackage[T2A]{fontenc}
\usepackage[utf8]{inputenc}
\usepackage[english, russian]{babel}

\usepackage[top=20mm, bottom=27mm, left=30mm, right=15mm]{geometry}

\linespread{1}

\newlength{\parinlen}
\setlength{\parinlen}{12.5mm}

\usepackage{indentfirst}
\setlength{\parindent}{\parinlen}    

\frenchspacing

\usepackage{perpage}
\MakePerPage{footnote}

\makeatletter 
    \def\@makefnmark{\hbox{\@textsuperscript{\normalfont\@thefnmark)}}}
\makeatother

\usepackage[bottom]{footmisc}

\makeatletter
    \renewcommand{\thesection}{\arabic{section}}
\makeatother

\setcounter{secnumdepth}{3}

\usepackage{tocloft}

\setlength{\cftbeforetoctitleskip}{-1em}
\setlength{\cftaftertoctitleskip}{1em}

\makeatletter
\renewcommand{\cftsecfillnum}[1]
{%
    \ifnum#1<10
    \gdef\@pnumwidth{6pt}
    \else
    \ifnum#1<100
    \gdef\@pnumwidth{14pt}
    \else
    \gdef\@pnumwidth{18pt}
    \fi\fi
    {\cftsecleader}\nobreak
    \makebox[\@pnumwidth][\cftpnumalign]{\cftsecpagefont #1}\cftsecafterpnum\par
}
\renewcommand{\cftsubsecfillnum}[1]
{%
    \ifnum#1<10
    \gdef\@pnumwidth{6pt}
    \else
    \ifnum#1<100
    \gdef\@pnumwidth{14pt}
    \else
    \gdef\@pnumwidth{18pt}
    \fi\fi
    {\cftsecleader}\nobreak
    \makebox[\@pnumwidth][\cftpnumalign]{\cftsecpagefont #1}\cftsecafterpnum\par
}
\makeatother

\renewcommand{\cftsecleader}{\cftdotfill{\cftsecdotsep}}
\renewcommand\cftsecdotsep{\cftdot}
\renewcommand\cftsubsecdotsep{\cftdot}

\cftsetindents{section}{0.5em}{1.0em}
\cftsetindents{subsection}{1.5em}{1.8em}

\usepackage{fancyhdr} 
\pagestyle{fancy}

\fancyhf{}
\fancyfoot[R]{\thepage} 
    \renewcommand{\footrulewidth}{0pt}
    \renewcommand{\headrulewidth}{0pt}
\fancypagestyle{plain}{\fancyhf{}\rfoot{\thepage}}

\makeatletter
    \renewcommand{\section}
    {
    \clearpage
        \@startsection{section}{1}
        {\parinlen}
        {-1em \@plus -1ex \@minus -.2ex}
        {1em \@plus .2ex}                
        {\raggedright\normalfont\normalsize\bfseries\MakeUppercase}
    }
\makeatother

\makeatletter
    \renewcommand{\subsection}
    {
        \@startsection{subsection}{2}
        {\parinlen}
        {-1em \@plus -1ex \@minus -.2ex} 
        {1em \@plus .2ex} 
        {\raggedright\normalfont\normalsize\bfseries}
    }
\makeatother 

\makeatletter
    \renewcommand{\subsubsection}{
    \@startsection{subsubsection}{3}
    {\parinlen}
    {-1em \@plus -1ex \@minus -.2ex}
    {\z@}                         
    {\raggedright\normalfont\normalsize\bfseries}}
\makeatother

\makeatletter
    \newcommand{\@csection}{
        \clearpage
        \@startsection{section}{1}
        {\z@}                                         
        {-1em \@plus -1ex \@minus -.2ex}
        {1em \@plus .2ex}
        {\centering\normalfont\normalsize\bfseries\MakeUppercase}}
        
    \newcommand{\csection}[1]{\@csection*{#1}\addcontentsline{toc}{section}{#1}} 
\makeatother % \large

\makeatletter
    \renewcommand{\@seccntformat}[1]{\csname the#1\endcsname{}~}
\makeatother

\usepackage[final]{graphicx}
\DeclareGraphicsExtensions{.pdf,.png,.jpg,.eps}
\graphicspath{{images/}}

\usepackage[nooneline]{caption}
\usepackage{subcaption}

\DeclareRobustCommand{\No}{\ifmmode{\nfss@text{\textnumero}}\else\textnumero\fi}

\DeclareCaptionLabelSeparator{stb}{~--~}

\DeclareCaptionLabelFormat{stbfigure}{Рисунок #2}
\DeclareCaptionLabelFormat{stbtable}{Таблица #2}

\captionsetup[figure]{labelformat=stbfigure, justification=centering, labelsep=stb, size=normalsize}
\captionsetup[table]{labelformat=stbtable, justification=raggedright, labelsep=stb, size=normalsize}

% \DeclareCaptionLabelFormat{stbsubfigure}{#2}
% \renewcommand{\thesubfigure}{\asbuk{subfigure}}
% \captionsetup[subfigure]{labelformat=stbsubfigure, justification=centering, size=normalsize}

\usepackage{calc}

\newlength{\wherelen} 
\settowidth{\wherelen}{где}

\newenvironment{explanation}
{
    \begin{itemize}[leftmargin=0cm, itemindent=\wherelen + \labelsep , labelsep=\labelsep]
    \renewcommand{\labelitemi}{}}
{
    \end{itemize}
}

\usepackage{amsmath}
\usepackage{amsfonts}
\usepackage{amssymb}
\usepackage{amsthm}
 
\usepackage{enumitem}

\makeatletter
    \AddEnumerateCounter{\asbuk}{\@asbuk}{щ)}
\makeatother

\setlist{nolistsep}

\renewcommand{\labelenumi}{\arabic{enumi}.}

\setlist[itemize,0]{itemindent=\parindent + 2.2ex, leftmargin=0ex, label=--}
\setlist[enumerate,1]{itemindent=\parindent + 2.7ex, leftmargin=0ex}
\setlist[enumerate,2]{itemindent=\parindent + \parindent - 2.7ex}

\AtBeginDocument{\numberwithin{equation}{section}}
\AtBeginDocument{\numberwithin{table}{section}}
\AtBeginDocument{\numberwithin{figure}{section}}

\usepackage{makecell}
\usepackage{multirow}
\usepackage{array}

% \usepackage{icomma}

% \usepackage{mathtext}

\makeatletter
    \renewcommand{\roman}[1]{\romannumeral #1}
    \renewcommand{\Roman}[1]{\expandafter\@slowromancap\romannumeral #1@}
\makeatother

\usepackage{longtable}
\usepackage{pdfpages}
\usepackage{textcomp, gensymb}
\usepackage{tabularx}

\def\hyph{-\penalty0\hskip0pt\relax}

\usepackage[square, numbers, sort&compress]{natbib}

\setlength{\bibsep}{0em}

\bibliographystyle{styles/belarus-specific-utf8gost780u}

\setlength\bibindent{-1.0900cm}

\makeatletter
\renewcommand\NAT@bibsetnum[1]{
    \settowidth\labelwidth{\@biblabel{#1}}
    \setlength{\leftmargin}{\bibindent}\addtolength{\leftmargin}{\dimexpr\labelwidth+\labelsep\relax}%
    \setlength{\itemindent}{-\bibindent+\parinlen-0.240cm}%
    \setlength{\listparindent}{\itemindent}
\setlength{\itemsep}{\bibsep}\setlength{\parsep}{\z@}%
    \ifNAT@openbib
        \addtolength{\leftmargin}{\bibindent}%
        \setlength{\itemindent}{-\bibindent}%
        \setlength{\listparindent}{\itemindent}%
        \setlength{\parsep}{10pt}%
    \fi
}

\mathchardef\mathequals=\mathcode`=
\begingroup\lccode`~=`=
  \lowercase{\endgroup\def~}{\mathequals\discretionary{}{\the\textfont0=}{}}
\AtBeginDocument{\mathcode`=="8000 }

\newcommand\z[1]{\ifhmode\unskip\fi\nobreak\ \discretionary{#1}{#1}{#1}\nobreak}