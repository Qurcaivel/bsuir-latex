% Основной стиль текста на странице

\documentclass[a4paper, oneside, 14pt, final]{extreport}

\usepackage[T2A]{fontenc}
\usepackage[utf8]{inputenc}
\usepackage[english, russian]{babel}

\usepackage[top=20mm, bottom=27mm, left=30mm, right=15mm]{geometry}

\linespread{1.15} % Recommended: 1.00, 1.15, 1.25

\newlength{\parinlen}
\setlength{\parinlen}{12.5mm} % Recommended: 1.25cm, 1.27cm

\usepackage{indentfirst}
\setlength{\parindent}{\parinlen}    

% Стиль сносок

\usepackage{perpage}
\MakePerPage{footnote}

\makeatletter 
\def\@makefnmark{\hbox{\@textsuperscript{\normalfont\@thefnmark)}}}
\makeatother

\usepackage[bottom]{footmisc}

% Стиль содержания

\usepackage{tocloft}

\setlength{\cftbeforetoctitleskip}{-1em}
\setlength{\cftaftertoctitleskip}{1em}

\makeatletter
\renewcommand{\cftsecfillnum}[1]
{%
    \ifnum#1<10%
    \gdef\@pnumwidth{6pt}%
    \else%
    \ifnum#1<100%
    \gdef\@pnumwidth{14pt}%
    \else%
    \gdef\@pnumwidth{18pt}%
    \fi\fi%
    {\cftsecleader}\nobreak%
    \makebox[\@pnumwidth][\cftpnumalign]{\cftsecpagefont #1}\cftsecafterpnum\par%
}

\renewcommand{\cftsubsecfillnum}[1]
{%
    \ifnum#1<10%
    \gdef\@pnumwidth{6pt}%
    \else%
    \ifnum#1<100%
    \gdef\@pnumwidth{14pt}%
    \else%
    \gdef\@pnumwidth{18pt}%
    \fi\fi%
    {\cftsecleader}\nobreak%
    \makebox[\@pnumwidth][\cftpnumalign]{\cftsubsecpagefont #1}\cftsubsecafterpnum\par%
}
\makeatother

\renewcommand{\cftsecleader}{\cftdotfill{\cftsecdotsep}}
\renewcommand{\cftsubsecleader}{\cftdotfill{\cftsubsecdotsep}}

\renewcommand\cftsecdotsep{\cftdot}
\renewcommand\cftsubsecdotsep{\cftdot}

\cftsetindents{section}{0.5em}{1.0em}
\cftsetindents{subsection}{1.5em}{1.8em}

% Стиль колонтитулов

\usepackage{fancyhdr} 
\pagestyle{fancy}

\fancyhf{}
\fancyfoot[R]{\thepage} 

\renewcommand{\footrulewidth}{0pt}
\renewcommand{\headrulewidth}{0pt}

\fancypagestyle{plain}{\fancyhf{}\rfoot{\thepage}}

% Стиль секций

\setcounter{secnumdepth}{3}

\renewcommand{\thesection}{\arabic{section}}

\makeatletter
\renewcommand{\@seccntformat}[1]{\csname the#1\endcsname{}~}

\renewcommand{\section}
{%
    \clearpage%
    \@startsection{section}{1}%
    {\parinlen}%
    {-1em \@plus -1ex \@minus -.2ex}%
    {1em \@plus .2ex}%  
    {\raggedright\normalfont\normalsize\bfseries\MakeUppercase}%
}

\renewcommand{\subsection}
{%
    \@startsection{subsection}{2}%
    {\parinlen}%
    {-1em \@plus -1ex \@minus -.2ex}%
    {1em \@plus .2ex}% 
    {\raggedright\normalfont\normalsize\bfseries}%
}

\renewcommand{\subsubsection}
{%
    \@startsection{subsubsection}{3}%
    {\parinlen}%
    {-1em \@plus -1ex \@minus -.2ex}%
    {1em \@plus .2ex}%                         
    {\raggedright\normalfont\normalsize\bfseries}%
}

\newcommand{\@csection}
{%
    \clearpage%
    \@startsection{section}{1}%
    {\z@}%                                   
    {-1em \@plus -1ex \@minus -.2ex}%
    {1em \@plus .2ex}%
    {\centering\normalfont\normalsize\bfseries}%
}

\newcommand{\csection}[1]
{%
    \@csection*{\MakeUppercase{#1}}%
    \addcontentsline{toc}{section}{#1}%
} 

\newcounter{appendix}
\renewcommand{\theappendix}{\Asbuk{appendix}}

\newcommand{\@appendix}
{%
    \clearpage%
    \@startsection{section}{1}%
    {\z@}%                                   
    {-1em \@plus -1ex \@minus -.2ex}%
    {1em \@plus .2ex}%
    {\centering\normalfont\normalsize\bfseries}%
}

\renewcommand{\appendix}[2]
{%
    \refstepcounter{appendix}%
    \@appendix*{\MakeUppercase{Приложение \theappendix} \\ (#1) \\ #2}%
    \markboth{\MakeUppercase{#1}}{}%
    \addcontentsline{toc}{section}{Приложение {\theappendix} (#1) #2}%
}
\makeatother

% Графические изображения

\usepackage[final]{graphicx}
\DeclareGraphicsExtensions{.pdf,.png,.jpg,.eps}
\graphicspath{{images/}}

% Стиль надписей к таблицам и изображениям

\usepackage[nooneline]{caption}
\usepackage{subcaption}

\DeclareCaptionLabelSeparator{stb}{~--~}

\DeclareCaptionLabelFormat{stbfigure}{Рисунок #2}
\DeclareCaptionLabelFormat{stbtable}{Таблица #2}

\captionsetup[figure]{labelformat=stbfigure, justification=centering, labelsep=stb, size=normalsize} % Recommended: size=small, size=normalsize
\captionsetup[table]{labelformat=stbtable, justification=raggedright, labelsep=stb, size=normalsize}

% Математика и оформление формул

\usepackage{calc}
\usepackage{amsmath}
\usepackage{amsfonts}
\usepackage{amssymb}
\usepackage{amsthm}

\newlength{\eqindent}
\setlength{\eqindent}{1.4em} % Recommended: baselineskip, 1.3em, 1.4em, 1.5em

\makeatletter
\expandafter\def\expandafter\normalsize\expandafter
{%
    \normalsize
    \setlength\abovedisplayskip{\eqindent - \baselineskip}
    \setlength\belowdisplayskip{\eqindent}
    \setlength\abovedisplayshortskip{\eqindent - \baselineskip}
    \setlength\belowdisplayshortskip{\eqindent}
}
\makeatother

\newenvironment{explanation}
{\begin{itemize}[leftmargin=0cm, itemindent=\widthof{где} + \labelsep , labelsep=\labelsep]\renewcommand{\labelitemi}{}}
{\end{itemize}}

% Нумерация и стиль перечислений

\usepackage{enumitem}

\makeatletter
    \AddEnumerateCounter{\asbuk}{\@asbuk}{щ)}
\makeatother

\setlist{nolistsep}

\renewcommand{\labelenumi}{\arabic{enumi}.}

\setlist[itemize,0]{itemindent=\parindent + 2.2ex, leftmargin=0ex, label=--}
\setlist[enumerate,1]{itemindent=\parindent + 2.7ex, leftmargin=0ex}
\setlist[enumerate,2]{itemindent=\parindent + \parindent - 2.7ex}

\AtBeginDocument{\numberwithin{equation}{section}}
\AtBeginDocument{\numberwithin{table}{section}}
\AtBeginDocument{\numberwithin{figure}{section}}

% Вспомогательные пакеты

\usepackage{makecell}
\usepackage{multirow}
\usepackage{array}
\usepackage{longtable}
\usepackage{pdfpages}
\usepackage{textcomp, gensymb}
\usepackage{tabularx}

% Стиль библиографии

\usepackage[square, numbers, sort&compress]{natbib}

\setlength{\bibsep}{0em}

\bibliographystyle{gost780u}

\setlength\bibindent{-1.0900cm}

\makeatletter
\renewcommand\NAT@bibsetnum[1]
{%
    \settowidth\labelwidth{\@biblabel{#1}}
    \setlength{\leftmargin}{\bibindent}\addtolength{\leftmargin}{\dimexpr\labelwidth+\labelsep\relax}%
    \setlength{\itemindent}{-\bibindent+\parinlen-0.240cm}%
    \setlength{\listparindent}{\itemindent}%
    \setlength{\itemsep}{\bibsep}\setlength{\parsep}{\z@}%
    
    \ifNAT@openbib
        \addtolength{\leftmargin}{\bibindent}%
        \setlength{\itemindent}{-\bibindent}%
        \setlength{\listparindent}{\itemindent}%
        \setlength{\parsep}{10pt}%
    \fi
}
\makeatother

% Поддержка ссылок с нижним подчеркиванием в BibTeX

\usepackage{url}
\urlstyle{same}

% Frenchspacing и перенос слов с дефисом

\def\hyph{-\penalty0\hskip0pt\relax}

\frenchspacing

% Гиперссылки в содержании и тексте

\usepackage[colorlinks=true, allcolors=black]{hyperref}